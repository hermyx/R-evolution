\section{Espèces}
Le joueur peut choisir une espèce qui appartient à un de ces règnes là : Animal, Végétal, Champignon ou Bactérien. Pour chacun des règnes, voici les différentes classes. Elles sont déterminées dynamiquement en fonction des stats.
\begin{itemize}
	\item Animal :\begin{itemize}
				\item Insecte
				\item Arachnide
				\item Crabe
				\item Poisson
				\item Amphibien
				\item Chélonien (tortues)
				\item Crocodilien
				\item Squamates (lézard/serpents)
				\item Oiseau
				\item Mammifère
			\end{itemize}
	\item Archaeplastida :\begin{itemize}
				\item 
			\end{itemize}
	\item Fungi :\begin{itemize}
				\item 
			\end{itemize}
	\item Prokaryota :\begin{itemize}
				\item 
			\end{itemize}
\end{itemize}
Une espèce se compose de ces attributs:
\begin{itemize}
	\item Energie : Energie de l'organisme. Celui-ci meurt s'il arrive à 0.
	\item Taille : Taille de l'organisme (en cm ?).
	\item Force : Force physique (musculaire) de l'espèce (en N).
	\item Vitesse : Vitesse de sprint de l'espèce (en m/s). Si cette valeur est égale à 0, cela veut dire que l'espèce ne peut pas se déplacer.
	\item Résistance : Résistance physique.
	\item Thermie : Résistance thermique (ectothermie/endotherme -- homéotherme/poïkilotherme).
	\item Sens : Tableau contenant les sens de l'espèce.
	\item Reproduction : Système de reproduction de l'espèce (spores, oeufs, mitose, etc).
	\item Métabolisme.
	\item Niveau trophique : régime alimentaire (insecte, plante, herbivore, carnivores, etc).
	\item Durée de vie : Fréquence de reproduction.
	\item Moment de vie : Si l'espèce est diurne ou nocturne.
	\item Rythme de vie : Toutes les espèces partiront du principe qu'elles suivent, de base, un rythme cicardien. C'est-à-dire, que leur durée d'activité est le jour, et qu'elles dorment la nuit, pour les espèces diurnes (inversement pour les nocturnes). Ce rythme peut-être modifié au cours du jeu.
	\item Résistance à la mutation : tout est dans le nom.
	\item Habiletés : Compétences spéciales (photosynthèse, régénation, camouflage, etc).
	\item Type d'environnement : air, terre ou mer.
	\item Environnement favori : type de tile préféré (montagne, etc.).
	\item Règne.
	\item Classe. 
\end{itemize}
Les attributs numériques sont des moyennes.