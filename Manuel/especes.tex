\section{Espèces}
Le joueur peut choisir une espèce qui appartient à un de ces règnes là : Animal, Végétal, Champignon ou Bactérien. Pour chacun des règnes, voici les différentes classes. Elles sont déterminées dynamiquement en fonction des stats.
\begin{itemize}
	\item Animal :\begin{itemize}
				\item Coléoptères (scarabés, lucioles, coccinelles, ...)
				\item Diptères (mouches, moustiques)
				\item Hymènoptères (abeilles, guêpes, fourmis)
				\item Lépidoptères (mites, papillons)
				\item Dictyoptères (blattes, termites, mantes)
				\item Orthoptères (sauterelles, criquets, grillons)
				\item Odonta (libellules)
				
				\item Mygalomorphes (mygales)
				\item Aranaeoorphes (araignées "classiques")
				\item Scorpiones (scorpions)
				
				\item Chéloniens (tortues)
				\item Crocodiliens (crocodiles, alligators)
				\item Serpentes (serpents)
				\item Autarchoglossa (lézards)
				\item Iguania (iguanes, chaméléons)
				
				\item Paléognathes (émeu, autruches)
				\item Sphenisciformes (manchots)
				\item Procellariiformes (oiseaux de mer)
				\item Accipitriformes (rapaces diurnes)
				\item Strigiformes (rapaces nocturnes)
				\item Psittaciformes (perroquets, perruches, ...)
				
				\item Marsuipiaux (kangourous, ...)
				\item Glires (rongeurs, lagomorphes)
				\item Primates (singes, lorisiformes, lémuriformes)
				\item Chiroptera (chauve-souris)
				\item Perissodactyla (rhinos, chevaux, zèbres, ...)
				\item Carnivora (félins, canins, ours, ...)
				\item Cetacea (dauphin, orque, baleine, ...)
				\item Ruminantia (cerfs, girafes, bovins)
				
				\item Lamniformes (requins)
				\item Rajiformes (raies)
				\item Anguiliformes (anguilles)
				\item Syngnathiformes (hippocampes)
				\item Perciformes (morue, dorade et autre poisson "commun")
				\item Coelacanthiformes (coelacanthes)
				\item Myctophiformes (poisson-lanternes)
				\item Characiformes (piranhas)
				
				\item Astacidea (homards, écrevisses, langoustines)
				\item Brachyura (crabes)
				\item Euphausiacea (krill (ou plancton))
				
				\item Urodèles (salamandres, tritons)
				\item Anoures (Grenouilles, crapauds)
			\end{itemize}
	\item Archaeplastida :\begin{itemize}
				\item 
			\end{itemize}
	\item Fungi :\begin{itemize}
				\item 
			\end{itemize}
	\item Prokaryota :\begin{itemize}
				\item 
			\end{itemize}
\end{itemize}
Une espèce se compose de ces attributs:
\begin{itemize}
	\item Energie : Energie de l'organisme. Celui-ci meurt s'il arrive à 0.
	\item Force : Force physique (musculaire) de l'espèce (en N).
	\item Agilité : Souplesse et capacité de manipulation du corps. Joue sur l'esquive.
	\item Vitesse : Vitesse maximale de sprint de l'espèce (en m/s). Si cette valeur est égale à 0, cela veut dire que l'espèce ne peut pas se déplacer.
	\item Accélération : Capacité de pouvoir monter rapidement en vitesse ou pas (en m/s.s).
	\item Résistance : Résistance physique.
	\item Thermie : Résistance thermique (ectothermie/endotherme -- homéotherme/poïkilotherme).
	\item Sens : Tableau contenant les sens de l'espèce.
	\item Reproduction : Système de reproduction de l'espèce (spores, oeufs, mitose, etc).
	\item Métabolisme.
	\item Morphologie.
	\item Niveau trophique : niveau dans la chaîne alimentaire (rien, autotrophe, herbivore, carnivore).
	\item Régime alimentaire : type de nourriture (bois, plante, insecte, viande fraîche, viande morte, ...).
	\item Durée de vie : Fréquence de reproduction.
	\item Moment de vie : Si l'espèce est diurne ou nocturne.
	\item Rythme de vie : Toutes les espèces partiront du principe qu'elles suivent, de base, un rythme cicardien. C'est-à-dire, que leur durée d'activité est le jour, et qu'elles dorment la nuit, pour les espèces diurnes (inversement pour les nocturnes). Ce rythme peut-être modifié au cours du jeu.
	\item Résistance à la mutation : tout est dans le nom.
	\item Habiletés : Compétences spéciales (photosynthèse, régénation, camouflage, etc).
	\item Type d'environnement : air, terre ou mer.
	\item Environnement favori : type de tile préféré (montagne, etc.).
	\item Règne.
	\item Classe. 
\end{itemize}
Les attributs numériques sont des moyennes.